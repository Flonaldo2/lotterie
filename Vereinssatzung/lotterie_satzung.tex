\documentclass{article}

\usepackage[ngerman]{babel}
\usepackage[utf8x]{inputenc}
\usepackage[T1]{fontenc}
\usepackage{enumerate}

\setlength{\textwidth}{15cm}
\setlength{\textheight}{24cm}
\addtolength{\topmargin}{-2cm}
\addtolength{\oddsidemargin}{-1.5cm}

\title{\textsf{\textbf{Vereinssatzung der Lotterie Hamburg e.V.}}}
\author{}
\date{}

\hyphenation{LOTTERIE}

\begin{document}
\maketitle

\begin{enumerate}[§ 1.]
\item \textsf{\textbf{Name, Sitz und Geschäftsjahr}}
\begin{enumerate}
\item Der Verein führt den Namen "Lotterie Hamburg e.V.".
\item Der Verein soll im Vereinsregister des Amtsgerichts Hamburg eingetragen werden.
\item Der Verein hat seinen Sitz in Hamburg. Die Adresse lautet Große Elbstraße 132, 22767 Hamburg.
\item Geschäftsjahr des Vereins ist das Kalenderjahr. Das erste Rumpfjahr endet am 31. Dezember 2018.
\end{enumerate}

\item \textsf{\textbf{Zweck}}
\begin{enumerate}
\item Zweck des Vereins ist die Förderung des Sports. Der Satzungszweck wird insbesondere durch das Anbieten sportlicher Übungen und die Förderung sportlicher Leistungen, die Veranstaltung von Wettkämpfen und durch die Teilnahme an Sportveranstaltungen verwirklicht.
\end{enumerate}

\item \textsf{\textbf{Wirtschaftlichkeit}}
\begin{enumerate}
\item Der Verein verfolgt ausschließlich und unmittelbar gemeinnützige Zwecke im Sinne des Abschnitts "`Steuerbegünstigte Zwecke"' der Abgabenordnung. Der Verein ist selbstlos tätig; er verfolgt nicht in erster Linie eigenwirtschaftliche Zwecke.
\item Mittel des Vereins dürfen nur für die satzungsmäßigen Zwecke verwendet werden. Die Mitglieder erhalten keine Zuwendungen aus Mitteln des Vereins. Es darf keine Person durch Ausgaben, die dem Zweck des Vereins fremd sind, oder durch unverhältnismäßig hohe Vergütungen begünstigt werden.
\end{enumerate}

\item \textsf{\textbf{Mitgliedschaft}}
\begin{enumerate}
\item Mitglied des Vereins kann werden:
\begin{enumerate}
\item jede natürliche Person,
\item eine juristische Person des öffentlichen oder privaten Rechts oder eine Handelsgesellschaft.
\end{enumerate}
\item Der Antrag für die Mitgliedschaft wird schriftlich vom Antragsteller gestellt. Über den Antrag entscheidet der Vorstand. Das Aufnahmegesuch eines Minderjährigen ist von dem/den gesetzlichen Vertretern zu stellen. Die Mitgliedschaft wird erworben durch die Aushändigung einer Satzung.
\item Die Mitgliedschaft endet:
\begin{enumerate}
\item mit dem Tod des Mitglieds,
\item durch schriftliche Austrittserklärung, gerichtet an ein Vorstandsmitglied,
\item durch Ausschluss aus dem Verein.
\end{enumerate}
\item Ein Mitglied, das in erheblichem Maß gegen die Vereinsinteressen verstoßen hat, kann durch Beschluss des Vorstands aus dem Verein ausgeschlossen werden. Vor dem Ausschluss ist das betroffene Mitglied persönlich oder schriftlich zu hören. Die Entscheidung über den Ausschluss ist schriftlich zu begründen und dem Mitglied mit Einschreiben zuzustellen.
\item Das Mitglied kann innerhalb einer Frist von einem Monat ab Zugang schriftlich Berufung beim Vorstand einlegen. Über die Berufung entscheidet dann die
Mitgliederversammlung.
Macht das Mitglied vom Recht der Berufung innerhalb der Frist keinen Gebrauch
unterwirft es sich dem Ausschließungsbeschluss.
\end{enumerate}

\item \textsf{\textbf{Organe}}
\begin{enumerate}
\item Die Organe des Vereins sind:
\begin{enumerate}
\item der Vorstand
\item die Mitgliederversammlung
\end{enumerate}
\end{enumerate}

\item \textsf{\textbf{Der Vorstand}}
\begin{enumerate}
\item Der Vorstand des Vereins besteht aus dem 1. Vorsitzenden, dem 2. Vorsitzenden, dem Kassenwart, dem Schriftführer, dem Sportwart und einem Beisitzer.
\item Der Verein wird gerichtlich und außergerichtlich jeweils einzeln durch den 1. und 2. Vorsitzenden vertreten.
\item Der Vorstand wird durch die Mitgliederversammlung auf die Dauer von zwei Jahren gewählt.
\item Er bleibt solange im Amt bis eine Neuwahl erfolgt. Scheidet ein Mitglied des Vorstandes
während der Amtsperiode aus, wählt der Vorstand ein Ersatzmitglied für den Rest der Amtsdauer des ausgeschiedenen Vorstandsmitgliedes.
\end{enumerate}

\item \textsf{\textbf{Die Mitgliederversammlung}}
\begin{enumerate}
\item Die Mitgliederversammlung ist jährlich vom 1. Vorsitzenden unter Einhaltung einer Einhaltungsfrist von 3 Wochen mittels einer Email an alle Mitglieder einzuberufen.
\item Dabei ist die vom Vorstand festgesetzte Tagesordnung mitzuteilen.
\item Die Mitgliederversammlung hat insbesondere folgende Aufgaben:
\begin{enumerate}
\item Genehmigung des Haushaltsplanes für das folgende Geschäftsjahr,
\item Entgegennahme des Rechenschaftsberichtes des Vorstands und dessen
Entlastung,
\item Festsetzung der Höhe des Mitgliedsbeitrages,
\item Beschlüsse über Satzungsänderungen und Vereinsauflösung,
\item Beschlüsse über die Berufung eines Mitglieds gegen seinen Ausschluss durch den
Vorstand.
\end{enumerate}
\item Der Vorstand hat unverzüglich eine Mitgliederversammlung einzuberufen, wenn das
Vereinsinteresse es erfordert oder ein Viertel der Mitglieder die Einberufung schriftlich und unter
Angaben der Gründe fordern.
\item Über die Beschlüsse der Mitgliederversammlung ist ein Protokoll aufzunehmen, das vom
Versammlungsleiter und dem Protokollführer zu unterzeichnen ist.
\end{enumerate}

\item \textsf{\textbf{Mitgliedsbeiträge}}
\begin{enumerate}
\item Von den Mitgliedern werden Beiträge erhoben. Für die Höhe der Mitgliedsbeiträge und Aufnahmegebühren ist die jeweils gültige Beitragsordnung maßgebend, die von der Mitgliederversammlung beschlossen wird.
\end{enumerate}

\item \textsf{\textbf{Auflösung des Vereins und Anfall des Vereinsvermögens}}
\begin{enumerate}
\item Bei Auflösung oder Aufhebung des Vereins oder bei Wegfall des steuerbegünstigten Zwecks fällt das Vermögen an eine juristische Person des öffentlichen Rechts oder eine steuerbegünstigte Körperschaft, die es unmittelbar und ausschließlich zur Förderung des Sports zu verwenden hat.
\end{enumerate}

\item \textsf{\textbf{Ordnungen}}
\begin{enumerate}
\item Ordnungen werden von der Mitgliederversammlung beschlossen und sind nicht Bestandteil dieser Satzung.
\item Für eine Änderung ist eine einfache Mehrheit ausreichend.
\item Der Verein hat folgende Ordnungen:
\begin{enumerate}
\item Geschäftsordnung
\item Beitragsordnung
\end{enumerate}
\end{enumerate}

\item \textsf{\textbf{Inkrafttreten}}
\begin{enumerate}
\item Die unveränderten Bestimmungen der Satzung entsprechen der Gründungssatzung vom 03.11.2018.
\end{enumerate}

\end{enumerate}
Ursprüngliche Fassung: 03.11.2018\\
Stand: 10.12.2018


\subsection*{Gründungsmitglieder}
\begin{enumerate}
\item Florian Lienkamp
\item Joachim Schreiber
\item Ole Basilon
\item Sören Matthies
\item Henri Schütte
\item Caroline Vogler
\item Moritz Kramer
\item Malte Groth
\item Felix Ruckdeschel
\item Christoph Marks
\item Vincent Göbel
\end{enumerate}

\end{document}
